\appendix
\section{Conventions and Notation\label{app_conventions}}


We loosely follow the conventions of \cite{korthikanti2022reducing}.  Common parameters:
\begin{itemize}
	\item $ A $: number of attention heads
	\item $ B $: microbatch size
	\item $ C $: compute (FLOPs)
	\item $ D $: the hidden dimension size
	\item $ E $: expansion factor for MLP layer (usually $ E=4 $)
	\item $ H $: $ D/A $, the head dimension size
	\item $ K $: the block size (maximum sequence length\footnote{In the absence of methods such as         ALiBi \cite{ALiBi}  can be used to extend the sequence length at inference time.})
	\item $ L $: number of transformer layers
	\item $ N _{ {\rm params}  } $: total number of model parameters
	\item $ N _{ {\rm ex}  } $: number of experts for MoE models.
	\item $ P $: pipeline parallel size
	\item $ S $: input sequence length
	\item $ T $: tensor parallel size
	\item $ V $: vocabulary size
	\item $ t $: various timescales
	\item $ p $: the precision of the elements of a tensor in bytes
	\item $ \lambda  $: various rates, e.g. $ \lambda _{ {\rm mem}  } $ is memory bandwidth
\end{itemize}
Where it makes sense, we try to use the lower-case versions of these characters to denote the
corresponding indices on various tensors. For instance, an input tensor with the above batch size,
sequence length, and vocabulary size would be written as $ x _{ bsv } $, with $ b \in \left \{ 0,
	\ldots, B - 1 \right \} $, $ s \in \left \{ 0, \ldots, S - 1\right \} $, and $  v \in \left \{ 0,
	\ldots, V -1\right \}$ in math notation, or as \mintinline{python}{x[b, s, v]} in code.


    Typical
transformers belong to the regime
\begin{gather}
	V \gg D, S \gg L, A \gg P, T \ .  \label{app_eq_transformers_approxs}
\end{gather}
For instance, GPT-2 and GPT-3 \cite{gpt2radford2019language, gpt3brown2020language} have $ V \sim
	\Ocal \left( 10 ^{ 4 } \right)  $, $ S, L \sim \Ocal \left( 10 ^{ 3 } \right)  $, $ L, A \sim \Ocal
	\left( 10 ^{ 2 } \right)  $. We will often assume also assume that\footnote{This condition ensures
	that the $ \Ocal \left( S ^{ 2 } \right)  $ FLOPs cost from self-attention is negligible
	compared to $ \Ocal \left( D ^{ 2 } \right)  $ contributions from other matrix multiplies.  It
	should be noted that in Summer 2023 we are steadily pushing into the regime where this condition
	does \textit{not}  hold.} $ S \lesssim D $ or the weaker\footnote{This condition ensures that the
	cost of reading the $ \Ocal \left( D ^{ 2 } \right)  $ weights is more than the cost of reading in
	the $ \Ocal \left( BSD \right)  $ entries of the intermediate representations.} $ BS \lesssim D $.

As indicated above,  we use zero-indexing. We also use \pyinline{python} code
throughout\footnote{Written in a style conducive to latex, e.g. no type-hints and clarity
prioritized over optimization.}  and write all ML code using standard \pyinline{torch} syntax. To
avoid needing to come up with new symbols in math expressions we will often use expressions like $ x
\leftarrow f(x) $ to refer to performing a computation on some argument ($ x $) and assigning the
result right back to the variable $ x $ again.

Physicists often joke (half-seriously) that Einstein's greatest contribution to physics was his
summation notation in which index-sums are implied by the presence of repeated indices and summation
symbols are entirely omitted. For instance, the dot product between two vectors would be written as
\begin{align}
	\vec{x} \cdot \vec{y} & = \sum _{ i } x _{ i } y _{ i } \equiv x _{ i } y _{  i }
	\label{app_eq_einstein_sum}
\end{align}
We use similar notation which is further adapted to the common element-wise deep-learning
operations.  The general rule is that if a repeated index appears on one side of an equation, but
not the other, then a sum is implied, but if the same index appears on both sides, then it's an
element-wise operation. The Hadamard-product between two matrices $ A $ and $ B $ is just
\begin{align}
	C _{ ij } & = A _{ ij } B _{ ij }\ .
\end{align}
Einstein notation also has implementations available for \pyinline{torch}:
\href{https://rockt.github.io/2018/04/30/einsum}{see this blog post on \pyinline{einsum}} or the
\href{https://einops.rocks/1-einops-basics/}{\pyinline{einops}} package.  We strive to write all
learnable weights in upper case.

In particular, we use \pyinline{einops} notation for concatenation and splitting: $ A _{ c } = A _{
    (de) }= B _{ de } $\footnote{The indexing is all row-major: if $ A _{ i } $ is $
        I$-dimensional, $ i \in \{0, \ldots, I-1\} $, then if we split this index as $ A _{ i } = A
        _{ (jk) } \equiv \bar{A} _{ jk } $, then the indices $ j, k $ will range over $ j \in \{0,
    \ldots , J\} $, $ k\in \{0, \ldots , K\} $ with $ I =J \times K $ and where numerically $ i = j
\times K + k $. More complex cases follow by induction.}. We will sometimes use a bar to indicate
tensors which  are derived from other tensors through such splitting operations, usually in the
context of tensor-sharding where devices only locally hold some shard of the tensor. In this
context, only some of the dimensions will be sharded across devices, and we may also put a bar over
the corresponding sharded index. For instance, consider a two-dimensional tensor $ M _{ ab } $ of
shape \pyinline{M.shape=(A, B)}: sharding this tensor across two devices across the final index
results in a tensor $ \bar{M}_{ a \bar{b} } $ which is of shape \pyinline{M_bar.shape=(A, B/2)} on
each device. As here, we will sometimes use bars to denote indices which are sharded over different
devices.

We also put explicit indices on operators such as Softmax to help clarify the relevant
dimension, e.g. we would write the softmax operation over the $ b $-index of some batched
tensor $ x _{ bvd\ldots } $ as
\begin{align}
	s _{ bvd\ldots } & = \frac{ e^{ x _{ bv d\ldots}  } }{ \sum _{ v = 0 } ^{  v= V-1 } e^{ x _{
						bvd\ldots } } } \equiv
	\Sm _{ v } \ x _{ bvd\ldots }
	\ , \label{app_eq_einstein_softmax}
\end{align}
indicating that the sum over the singled-out $ v $-index is gives unity.

\section{Collective Communications \label{app_collective_communications} }

A quick refresher on common distributed
\href{https://docs.nvidia.com/deeplearning/nccl/user-guide/docs/usage/collectives.html}{communication
    primitives}.  Consider $ R $ ranks with tensor data $ x ^{ (r) }  $ of some arbitrary shape
    \pyinline{x.shape}, which takes up $ M $ bytes of memory, where $ r $ labels the worker and any
    indices on the data are suppressed. For collectives which perform an operation over a specific
    dimension, the \pyinline{torch} convention is that it operates over \pyinline{dim=0}. The $ r=0
    $ worker is arbitrarily denoted the \textit{chief}. Some operations are easiest to describe by
    forming the logical super-tensor \pyinline{X=torch.stack([x0, x1, ...], dim=0} of shape
    \pyinline{X.shape==Size(R, ...)} such that the tensor on rank \pyinline{r} is \pyinline{x=X[r]}.
    Then, the primitive operations are:
\begin{itemize}
	\item \pyinline{Broadcast}: all workers receive  the chief's data, $ x ^{ (0) }  $.
	\item \pyinline{Gather}: all workers communicate their data $ x _{ n } $ to the chief, e.g. in a
	      concatenated array $ [x ^{ 0 }, x ^{ 1 }, \ldots , x ^{ R-1 }] $. E.g., the chief gets
          \pyinline{x_out = X.reshape(R*X.shape[1], X.shape[2:])}.
	\item \pyinline{Reduce}: data is \pyinline{Gather}-ed to the chief, which then performs some
	      operation (\pyinline{sum}, \pyinline{max}, \pyinline{concatenate}, etc.) producing a new tensor $
		      x' $ on the chief worker. E.g., for \pyinline{sum} the chief gets
              \pyinline{x_out = X.sum(dim=0)}.
    \item \pyinline{ReduceScatter}: a reducing operation (e.g. \pyinline{sum}) is applied to the $ x
        ^{ (r) } $ to produce a $ x' $ of the same shape (e.g. $ x'= \sum x ^{ (r) } $) and each
        worker only receives a $ 1/R $ slice (and hence $ M/R $ byte) of the result\footnote{Note that \pyinline{AllGather}
            and \pyinline{ReduceScatter} are morally conjugate to each other. In the former, each
            worker ends up with $ R $ times as much data as they started with, while in
        \pyinline{ReduceScatter} they end up with $ 1/R $ of their initial data. One is nearly a
        time-reversed version of the other, which is a way of remembering that they have the came
        communication cost. They also compose to produce an output of the same initial size, as in
        \pyinline{AllReduce}.}. A ring implementation sends $ M \times \frac{ R-1 }{ R } $ bytes
        over each link in the ring. E.g., for \pyinline{sum} rank \pyinline{r} gets output
        \pyinline{x_out = X.sum(dim=0).tensor_split(R, dim=0)[r]}.
    \item \pyinline{AllGather}: all data $ x ^{ (r) } $ is communicated to all workers; each worker
        ends up with the array $ [x ^{ 0 }, x ^{ 1 }, \ldots , x ^{ R-1 }] $. Functionally
        equivalent to a \pyinline{Gather} followed by \pyinline{Broadcast}. A ring implementation
        sends $ M \times \left ( R-1 \right ) $ bytes over each link in the ring. E.g., all ranks
        get \pyinline{x_out = X.reshape(R*X.shape[1], X.shape[2:])}.
	\item \pyinline{AllReduce}: all workers receive the same tensor $ x' $ produced by operating on
	      the $ x ^{ (r) } $ with \pyinline{sum}, \pyinline{mean}, etc. Functionally equivalent to a
	      \pyinline{Reduce} followed by \pyinline{Broadcast}, or a \pyinline{ReduceScatter} followed
	      by a \pyinline{AllGather} (the more efficient choice\footnote{The former strategy scales
		      linearly with the number of worker, while the latter strategy underlies ``ring"
		      \pyinline{AllReduce} which is (nearly) independent of the number of workers: if each
		      worker carries data of size $ D $ which is to be \pyinline{AllReduce}-d, a total of $
			      \frac{ 2 \left ( R-1 \right )D }{ R } $ elements need to be passed around.
		      \href{https://andrew.gibiansky.com/blog/machine-learning/baidu-allreduce/}{See this blog
			      post for a nice visualization} or \cite{bandwidthOptimalAllReduce2009} for a relevant
              paper.\label{foot_all_reduce}}). In the latter case, the total cost is $ 2M \times
              \frac{ R-1 }{ R } $, due to \pyinline{AllReduce}-ing the initial $ M $-sized data, and
              then \pyinline{AllGather}-ing the $ M/R $-sized reductions. E.g., for \pyinline{sum}
              all ranks get \pyinline{x_out = X.sum(dim=0)}.
      \item  \pyinline{Scatter}: One worker gives shards of a tensor to all workers. If the worker
          is scattering tensor $ T _{ x } $ over the given index, a \pyinline{Scatter}
          effectively shards this as $ T _{ x } \longrightarrow T _{ (\bar{r}y) } $, each worker
          getting a $ \bar{r} $-shard. If \pyinline{x} is the chief's data, rank \pyinline{r}
          receives \pyinline{x_out = x.tensor_split(R, dim=0)[r]}.
      \item  \pyinline{AllToAll}: All workers receive shards of all others worker's tensors.  If
          every worker has a tensor $ T _{ \bar{r}y } $, for one value of $ \bar{r} $, which we
          imagine came from a sharding a tensor $ T _{ x } = T _{ (\bar{r}y) } $, then an
          \pyinline{AllToAll} over the $ y $ index produces produces the tensor $ T _{ z \bar{r} }$
          defined by $T _{ z \bar{r} }= T _{ x } $ on all workers. E.g. rank \pyinline{r} receives
          \pyinline{x_out = X.reshape(X.shape[1], R, X.shape[:2])[:,r]}.
\end{itemize}



\section{Hardware}

Basic information about relevant hardware considerations. Much of the following is from the
\href{https://docs.nvidia.com/deeplearning/performance/dl-performance-gpu-background/index.html}{NVIDIA
	docs}.



\subsection{NVIDIA GPU Architecture}

NVIDIA GPUs consist of some amount of relatively-slow off-chip DRAM memory\footnote{This is the number usually reported when
	discussing a given GPU, e.g. 32GiB for the top-of-the-line A100}, relatively-fast on-chip SRAM, and
a number of \textbf{streaming multiprocessors} (SMs) which perform the parallel computations.  Inside
more-recent GPUs, the SMs carry both ``CUDA cores" and "Tensor cores", where the latter are used for matrix-mulitiplies
and the former for everything else.


A few numbers of primary importance:
\begin{itemize}
	\item The rate at which data can be transferred from DRAM to SRAM ($ \lambda _{ {\rm mem} } $)
	\item The number of FLOP/s, which is more fundamentally computed by multiplying the number of
	      SMs by the FLOPS/cycle of each SM for the specific operation under consideration (see the NVIDIA
	      docs) by the clock rate: $ N _{ {\rm SM} }\cdot  \lambda _{ {\rm FLOPs/cycle} }\cdot  \lambda _{
			      {\rm clock} } $
\end{itemize}

The terminology and structure of the memory hierarchy is also important to understand. Types of
memory, from slowest to fastest:
\begin{itemize}
	\item \textbf{Global} memory is the slow, but plentiful, off-chip DRAM. It is the type of memory
	      typically used as kernel arguments
	\item \textbf{Constant} memory is read only and accessible by all threads in a given block. The
	      size of arrays in constant memory must be known at compile time
	\item \textbf{Local Memory} is similarly slow to global memory, but more plentiful than register
	      memory, and privately to individual threads and is allocated from within a kernel. When
	      registers run out, local memory fills the gap
	\item \textbf{Shared} memory is shared between all threads in a given block. Shared memory is
	      effectively a user-controlled cache. The size of arrays in shared memory must  be known at
	      compile time
	\item \textbf{Registers} hold scalar values and small tensors whose values are known at compile
	      time. They are local to each thread and they are plentiful since each thread needs its own
	      set of registers: 65,536 = $ 2 ^{ 16 } $ registers per SM an A100.
\end{itemize}
\href{https://www.youtube.com/watch?v=QQceTDjA4f4&t=2124s}{An excellent video overview of CUDA and NVIDIA GPU architecture which covers some of the above is here.}

\subsection{CUDA Programming Model}

The CUDA programming model uses a hierarchy of concepts:
\begin{itemize}
    \item \textbf{Threads}  are the fundamental unit of execution\footnote{Threads are always
        physically launched in \textbf{Warps} which consist of 32 threads.} which each run the same
        CUDA \textbf{Kernel}, or function, on different data inputs in parallel. Threads within the
        same block (below) may share resources, like memory, and may communicate with each other.
        Individual threads are indexed through the \textbf{threadIdx} variable, which has
        \cudainline{threadIdx.{x, y, z}} attributes with \cudainline{threadIdx.x} in
        \cudainline{0, ..., blockDim.x -1} and similar.

	\item Threads (and hence warps) are organized into 3D \textbf{blocks}. The size and indices of
	      the blocks can be accessed through the \cudainline{blockDim} and \cudainline{blockIdx}
	      variables, respectively, with \cudainline{blockIdx.x} in \cudainline{0, ..., gridDim.x - 1}.
	      \cudainline{blockDim.x *blockDim.y * blockDim.z} total threads run in a block.
	\item Blocks are organized into 3D \textbf{groups}. The size of the gird dimensions can be
	      accessed through the \cudainline{gridDim} variable, with similar attributes to the above. \\
	      \cudainline{gridDim.x * gridDim.y * gridDim.z} total blocks run in a grid.
\end{itemize}

The number of threads which can be launched in a given block is hardware limited; A100 80GiB GPUs
can run up to 1024 threads in a SM at a time (32 blocks with 32 threads each), for instance. Hence,
block and grid sizes need to be adjusted to match the problem size. There are also important memory
access considerations here. The 1024 threads which can be launched can also read sequentially from
memory and efficient usage implies that choosing the block size such that we are doing these reads
as often as possible is ideal.

\subsection{NVIDIA GPU Stats \label{app_gpu_stats}}


Summary of some relevant NVIDIA GPU statistics:
\begin{center}
	\begin{tabular}{| c |c| c |c |c| c| }\hline
		GPU                                                                                                                                  & Memory & $ \lambda _{ {\rm FLOP/s }  } $ & $ \lambda _{ {\rm mem}  } $ & $ \lambda _{ {\rm math}  } $ & $ \lambda _{ {\rm comms}  } $ \\ \hline
		\href{https://www.nvidia.com/content/dam/en-zz/Solutions/Data-Center/a100/pdf/nvidia-a100-datasheet-nvidia-us-2188504-web.pdf}{A100} & 80GiB  & 312  TFLOP/s                    & 2.0 TiB/s                   & 156 FLOPS/B                  & 300 GiB/s                     \\ \hline
		\href{https://www.nvidia.com/content/dam/en-zz/Solutions/Data-Center/a100/pdf/nvidia-a100-datasheet.pdf}{A100}                       & 40GiB  & 312  TFLOP/s                    & 1.6 TiB/s                   & 195 FLOPS/B                  & 300 GiB/s                     \\ \hline
		\href{https://images.nvidia.com/content/technologies/volta/pdf/volta-v100-datasheet-update-us-1165301-r5.pdf}{V100}                  & 32GiB  & 130  TFLOP/s                    & 1.1 TiB/s                   & 118  FLOPS/B                 & 16 GiB/s                      \\ \hline
	\end{tabular}
\end{center}
where
\begin{itemize}
	\item $ \lambda _{ {\rm  FLOP/s}  } $ is flops bandwidth (for \pyinline{float16/bfloat16}  multiply-accumulate ops)
	\item $ \lambda _{ {\rm  mem}  } $ is memory bandwidth
	\item $ \lambda _{ {\rm  math}  } = \frac{  \lambda _{ {\rm FLOP/s}  } }{ \lambda _{ {\rm mem} } } $ is \href{https://docs.nvidia.com/deeplearning/performance/dl-performance-gpu-background/index.html#gpu-arch}{math bandwidth}
	\item $ \lambda _{ {\rm  comms}  } $ is one-way communication bandwidth
\end{itemize}
A useful approximate conversion rate is that $ 1 \ {\rm TFLOP/s} \approx 100\  {\rm PFLOP/day} $.

Important practical note: the $ \lambda _{ {\rm FLOP/s} } $ numbers should be taken as aspirational.
Out-of-the box, \pyinline{torch.float16} matrix-multiplies in \pyinline{torch} with well-chosen
dimensions tops out around $ \sim 250 $ FLOPS/s




\section{Compute-bound vs Memory-bound \label{app_compute_mem_bound} }

If your matrix-multiplies are not sufficiently large on, you are wasting resources
\cite{he2022brrrrfromfirstprinciples}. The relevant parameters which determine sufficiency are $
	\lambda _{ {\rm FLOP/s}  } $ and $ \lambda _{ {\rm mem}  } $, the FLOPs and memory bandwidth,
respectively. The ratio $  \lambda _{ {\rm math}  } \equiv \frac{  \lambda _{ {\rm FLOP/s}  } }{
		\lambda _{ {\rm mem}  } }   $ determines how many FLOPS you must perform for each byte loaded from
memory; see App.~\ref{app_gpu_stats}. If your computations have a FLOPs/B ratio which is larger than
$ \lambda _{ {\rm math}  } $, then you are compute-bound (which is good, as you're maximizing
compute), and otherwise you are memory(-bandwidth)-bound (which is bad, since your compute
capabilities are idling). The FLOPs/B ratio of your computation is sometimes called the
\textbf{compute intensity} or \textbf{arithmetic intensity}. When compute bound, a process takes
time $ \sim F/\lambda _{ {\rm FLOP/s} } $, while memory-bound processes take time\footnote{Note that
	the time is not additive, e.g. compute-bound tasks do not take time $ \sim F/\lambda _{ {\rm FLOP/s}
		} +M/\lambda _{ {\rm mem} }$ because they are not sequential: compute and memory-communications can
	be concurrent.} $ \sim M/\lambda _{ {\rm mem} } $.



\subsection{Matrix-Multiplications vs. Element-wise Operations}

For instance, to multiply a \pyinline{(B, S, D)}-shaped tensor $ z _{ bsd } $ by a \pyinline{(D, D)}-shaped
weight-matrix $ W _{ d d' } $, $ p \left ( BDS +D ^{ 2 } \right ) $ bytes must be transferred from
DRAM to SRAM at a rate $ \lambda _{ {\rm  mem}  } $, after which we perform $ 2BSD ^{ 2 } $ FLOPs,
and write the \pyinline{(B, S, D)}-shaped  result back to DRAM again, for a ratio of
\begin{align}
	\frac{ 1 }{ p } \frac{ BDS }{ 2BS + D } \ \left ( {\rm FLOPs/B} \right )  \ .
\end{align}
We want to compare this against $ \lambda _{ {\rm math}  } $, which from
App.~\ref{app_gpu_stats} we take to be $ \Ocal \left( 100\  {\rm FLOPs/B} \right)  $, and plugging
in any realistic numbers, shows that such matrix-multiplies are essentially always compute-bound.
Compare this to the case of some element-wise operation applied to the same $ z _{ bsd } $ tensor
whose FLOPs requirements are $ \sim C\times BDS $ for some constant-factor $ C \ll S, D $.  Then,
then FLOPS-to-bytes ratio is $ \sim \frac{ C }{ p } $, which is \textit{always} memory-bound for
realistic values of $ C $. The moral is to try and maximize the number of matrix-multiplies and
remove as many element-wise operations that you can get away with.


\subsection{Training vs. Inference}

Finally, we note that the above has implications for the Transformers architecture as a whole, and
in particular it highlights the difficulties in efficient inference. Under the assumptions of
Sec.~\ref{sec_flops_training}, $ \sim \Ocal \left( BSN _{ {\rm params}  }  \right)  $
total FLOPs needed during training, while the number of bytes loaded from and written to memory are
$ \Ocal \left( BDLS + N _{ {\rm params}  } \right)  \sim \Ocal \left( \frac{ BS N _{ {\rm params}  } }{ D}+ N _{ {\rm params}  } \right)  $
which is $ \Ocal \left( N _{ {\rm  params}  } \right)  $ for not-super-long sequence lengths.  The
arithmetic intensity is therefore $ \Ocal \left( BS \right)  $ and so training is compute-bound in any
usual scenario, even at small $B \sim \Ocal \left( 1 \right)  $ batch sizes (as long as individual operations in the network don't suffer from outlandish
memory-boundedness). The problem during inference is that (if using the kv-cache; see
Sec.~\ref{sec_kv_cache}) we only need to process a \textit{single} token at a time and so $ S
	\longrightarrow 1 $ in the numerator in the preceding, while the denominator is also weighed down by
the  kv-cache in the attention layers.

In more detail, the \pyinline{MLP} layers just process $ S=1 $ length tensors during generation, but
are insensitive to the kv-cache, so their intensity comes from just setting $ S=1 $ in the above,
\begin{align}
	\sim \frac{ BD  }{ B + D } \ ,
\end{align}
dropping $\Ocal \left( 1 \right)  $ factors now, while the attention layers have a ratio of the form
\begin{align}
	\sim \frac{ BDS+BD ^{ 2 } }{ BD + D ^{ 2 }+BDS }\ ,
\end{align}
where the last term in the denominator is due to the cache. Now at small $B \sim \Ocal \left( 1
	\right)  $ batch sizes, both intensities reduce to $ \Ocal \left( B \right)  $, which is
insufficient to be compute-bound.  In the large $ B \gtrsim D/S $ limit, they at least become $ \Ocal \left( D
	\right)  $ and $ \Ocal \left( 1 + \frac{ D }{ S } \right)  $, respectively, which may be enough to be
compute-bound, but it's hard to even get into this regime. Note, the importance of the ratio $ D/S
$. The hidden dimension fixes the context length scale at which inference can never be
compute-bound, in the absence of additional tricks not considered here\footnote{One such trick: the
	multi-query attention of Sec.~\ref{subsec_multi_query_attn} improves everything a factor of $
		A $: the large batch regime is $ B \gtrsim \frac{ D }{ AS }$ and the intensity ratio becomes
	$ \Ocal \left( 1 + \frac{ D }{ AS } \right)  $. An analysis equivalent to the one performed here can be found in the original paper
	\cite{shazeer2019fast}.}.

\subsection{Intra- and Inter-Node Communication}

For intra-node communication, GPUs are connected by either PCIe or NVLink, generally.
\begin{itemize}
	\item \href{https://blogs.nvidia.com/blog/2023/03/06/what-is-nvidia-nvlink/}{NVLink}
	      interconnects are continually updated and achieve speeds of $ \lambda _{ {\rm comm} } ^{
			      {\rm intra} } \sim 300$ GiB/s.
\end{itemize}


For inter-node communication, nodes are often connected by:
\begin{itemize}
	\item \href{https://en.wikipedia.org/wiki/InfiniBand}{InfiniBand} apparently also achieves speeds $ \lambda _{ {\rm comm} } ^{
			      {\rm intra} } \sim 100$ GiB/s?  Haven't found a clear reference. But in any case, the
	      bandwidth is divided amongst the GPUs in the node, leading to a reduction by $ \sim 8 $.
\end{itemize}

\section{Batch Size, Compute, and Training Time \label{app_batch_size}}

The amount of compute directly determines the training time, but not all ways of spending compute
are equivalent. We follow the discussion in \cite{mccandlish2018empirical} which gives a rule of thumb
for determining the optimal batch size which is sometimes used in practice. The basic point is that
all of the optimization steps take the gradient $ \bfg  $ as an input, and since the gradient is the
average over randomly selected datapoints, steps are more precise as the batch size increases (with
diminishing returns, past a certain point, but the computational cost also rises with batch size,
and a balance between the two concerns should be struck.

Consider vanilla SGD and study how the training loss changes with each step.  We randomly sample $ B
$ datapoints $ x \in \Dcal$ from the dataset through some i.i.d. process\footnote{The below uses
	sampling with replacement, while in practice we sample without replacement, but the different is
	negligible for all practical cases.}. Each corresponding gradient $ \bfg (x) = \partial _{ w} \Lcal
	\left ( w, x \right ) $ is itself a random variable whose average is the true gradient across the
entire dataset $ \bar{\bfg} $ and we take the variance to be
\begin{align}
	{\rm Var}[\bfg (x), \bfg (x')] & =\Sigma
\end{align}
for some matrix $ \Sigma  $ with (supressed) indices spanning the space of model weights. Taking instead the mean
of a sum of such estimates, $ \bfg _{ B }\equiv \frac{ 1 }{ B } \sum _{ x \in \Bcal} \bfg (x) $,
the mean stays the same, but the variance reduces in the usual way: $ {\rm Var}[\bfg _{ B } (x),
			\bfg _{ B }(x')]=\Sigma /B$.

Study the mean loss across the entire dataset: $ \Lcal \left ( w \right ) = \langle
	\Lcal \left ( w, x \right) \rangle$. Using SGD we take a step $ w \longrightarrow w -\eta \bfg _{ B }   $
and change the loss as
\begin{align}
	\Lcal \left ( w -\eta \bfg _{ B } \right )  = \Lcal \left ( w  \right ) -\eta \bar{\bfg }\cdot
	\bfg _{ B } + \frac{ 1 }{ 2 } \bfg _{ B }\cdot  H \cdot  \bfg _{ B } + \Ocal \left( \bfg _{ B }
	^{ 3 } \right)\ ,
\end{align}
where $ H $ is the true hessian of the loss over the entire dataset at this value of the weights.
Taking the expectation value and minimizing the results  over $ \eta  $ gives the optimal choice:
\begin{align}
	\eta _{ \star } & = \frac{ \eta _{ {\rm  max} } }{ 1 + \frac{ B _{ {\rm  noise} } }{ B } }\ , \quad  \eta _{ {\rm  max} }\equiv \frac{ \bar{\bfg } ^{ 2 } }{ \bar{\bfg }\cdot  H \cdot  \bar{\bfg } } , \quad  B _{ {\rm  noise} } \equiv \frac{ \Tr H \cdot \Sigma  }{  \bar{\bfg }\cdot  H \cdot  \bar{\bfg }}\ .
\end{align}
Notably, the above supports the usual rule of thumb that the learning rate should be increased
proportionally to the batch size, at least whenever $ B \ll B _{ {\rm  noise} } $. The diminishing
returns of pushing batch sizes past $ B _{ {\rm noise} } $ are also evident. In practice it is too
expensive to compute the Hessian, but thankfully the entirely unjustified approximation in which the
Hessian is multiple of the identity such that
\begin{align}
	B _{ {\rm noise} } & \approx B _{ {\rm  simple} } \equiv \frac{ \Tr \Sigma  }{ \bar{\bfg } ^{ 2 } }\ ,
\end{align}
is somehow a decent approximation empirically, and an estimator can be created for $ B _{ {\rm noise} } $
in a data-parallel setup; see \cite{mccandlish2018empirical} or
\href{https://github.com/crowsonkb/k-diffusion/blob/ab527a9a6d347f364e3d185ba6d714e22d80cb3c/k_diffusion/gns.py#L1}{Katherine
	Crowson's implementation} or \href{https://github.com/EleutherAI/gpt-neox/blob/408e29d9c746a02d842917bb7447c5c4be0b42d4/megatron/gradient_noise_scale/gradient_noise_scale.py#L1}{neox} for more.


We can further characterize the trade-off between compute and optimization steps. The expected
decrease in loss per update is then
\begin{align}
	\langle \delta \Lcal \rangle & \approx \frac{ \eta _{ {\rm  max} } }{ 1 + \frac{ B _{ {\rm  noise} } }{ B } } \bar{\bfg } ^{2 } + \Ocal \left( \eta _{ {\rm max} } ^{ 2 } \right) \ ,
\end{align}
that is, we would need $ 1 + \frac{ B _{ {\rm noise} } }{ B } $ times as many SGD steps to make the
same progress we would have as compared to full-batch SGD. If $ S _{ {\rm  min} } $ is the number of
steps that would have been needed for full-batch SGD, we would need $ S=S _{ {\rm  min} } + S _{
			{\rm  min} } \frac{ B _{ {\rm noise} } }{ B } $ steps for minibatch SGD.  The total number of
examples seen is correspondingly $ E = S _{ {\rm  min} } \times  \left ( B _{ {\rm  noise} } +B
	\right)\equiv E _{ {\rm min} }+ S _{ {\rm  min} }B $, and so we see the trade-off between SGD steps
$ S $ and compute $ E $ alluded to above.  These relations can be written as\footnote{The analysis
	here is simplified in that it assumes that the noise scale and the chosen batch size are both
	time-independent. There is confusing logic treating the more general case where both
	$ B _{ {\rm noise} } $ and $ B $ vary with step in \cite{mccandlish2018empirical}, but in any case,
	the ultimate relations they use are effectively the same.}
\begin{align}
	\left ( \frac{ S }{ S _{ {\rm  min} } }-1 \right )\left ( \frac{ E }{ E _{ {\rm min} } }-1 \right ) & =1
\end{align}
which represent hyperbolic Pareto frontier curves. So, solutions are of the form $ S=\left ( \alpha
	+1 \right )S _{ {\rm  min} } $, $ E=\left ( \frac{ 1 }{ \alpha  } +1 \right )E _{ {\rm  min} } $
and since $E=BS $ the corresponding batch size is $ B _{ {\rm crit} } \equiv  \frac{ 1 }{ \alpha  }
	B _{ {\rm noise} }$. The parameter $ \alpha $ characterizes how much you value the trade-off between
these two factors and a reasonable balance is the $ \alpha =1 $ solution for which $S = 2S _{ {\rm
					min} } $, $ E=2E _{ {\rm min} } $ and $ B _{ {\rm crit} }= B _{ {\rm  noise} } $ exactly.

Correspondingly, in \cite{mccandlish2018empirical} they suggest training at precisely this batch
size.  But it seems much more relevant to balance time against compute directly, rather than
optimization steps vs compute. Modeling the total training time by $ T\approx S \left ( \kappa B + \sigma  \right)$
for some $ \kappa ,\sigma  $ to model compute costs\footnote{Computation and communication costs
	each scale with $ B $, the optimizer step does not (and maybe some overhead?), for instance.}, then
the above is equivalent to
\begin{align}
	T & = \frac{\left ( E _{ {\rm min} } + S _{ {\rm min} } B \right) \left ( \kappa B+ \sigma   \right ) }{  B  }\ .
\end{align}
which has a minimum at
\begin{align}
	B & = \sqrt{\frac{ \sigma E _{ {\rm min} } }{ \kappa S _{ {\rm min} }   }}\ .
\end{align}
for which the total time is
\begin{align}
	T _{ {\rm  min} } & = \left ( \sqrt{\kappa E _{ {\rm  min} }} - \sqrt{\sigma S _{ {\rm min} }} \right ) ^{ 2 }\ .
\end{align}
In comparison, the total time for the $ B _{ {\rm  crit} } = \frac{ E _{ {\rm min} } }{ S _{ {\rm
						min} } } $ strategy of \cite{mccandlish2018empirical} gives $ T _{ {\rm  min} } = 2 \left ( \kappa E
	_{ {\rm min} } + \sigma S _{ {\rm min} } \right ) $ which is a factor of $ \frac{ 2 }{ 1- \frac{
			\sqrt{\sigma \kappa B _{ {\rm noise} } } }{ \kappa B _{ {\rm noise}  } + \sigma } } $ larger.
So, this seems like a better choice of optimal batch size, if you value your time.


\section{Cheat Sheet \label{app_cheat_sheet}}

Collecting all of the most fundamental equations, given to various degrees of accuracy.

Number of model parameters:
\begin{align}
	N _{ {\rm params} } & =   (4+2E)LD ^{ 2 } + VD+ \Ocal \left( DL \right) \approx   \left ( 4+2E \right )LD ^{ 2 }\ ,
\end{align}
assuming no sharding of the embedding matrix.


\paragraph{Training}

Memory costs for mixed-precision training:
\begin{align}
	M _{ {\rm model} }                   & =p _{ {\rm model} } N _{ {\rm params} } \nn
	M _{ {\rm optim} }                   & =  \left ( s _{ {\rm states} }+1 \right) \times p _{ {\rm master } } N _{ {\rm params} } \nn
	M _{ {\rm act}  } ^{ {\rm  total}  } & =\frac{ 2BDLS   \left ( p(E+4) + 1   \right ) }{ T }
	+ \frac{ ABLS ^{ 2 } \left ( 2p+1\right ) }{ T }  + \Ocal \left( BSV \right)
\end{align}
where $ s _{ {\rm  states} } $ is the number of optimizer states, e.g. $ s=0 $ for SGD and $ s=2 $
for Adam. FLOPs total:
\begin{align}
	F _{ {\rm total}  } ^{ {\rm  model}  } & \approx 12 BDLS \left ( S + \left ( 2+E \right )D \right ) .
\end{align}


