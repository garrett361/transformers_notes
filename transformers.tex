\documentclass[11pt]{article}
%Importing custom commands
\usepackage{latex_goon}
\title{Transformers}
\author{Garrett Goon}
\begin{document}
%
%\maketitle

\vspace{1truecm}
%
%
\renewcommand{\thefootnote}{\fnsymbol{footnote}}
\begin{center}
{\huge \bf{Transformers}}
\end{center}


\begin{abstract}

Notes on various aspects of Transfomers.

\end{abstract}

\tableofcontents


\renewcommand*{\thefootnote}{\arabic{footnote}}
\setcounter{footnote}{0}

\section{To-Do List}

\begin{itemize}
\item
\end{itemize}


\section{Non-Equilibirium EFTs\label{sec:NonEqEFTs}}




We review here the general formalism for studying non-equilibrium effective field theories (EFTs) in the manner proposed in \cite{Glorioso:2018wxw}.


\subsection{Rough Outline\label{sec:Outline}}

A rough outline of the physical picture is as follows:
\begin{itemize}
\item In generic out-of-vacuum systems, things are messy.  In particular, fluctuations create and destroy quantities not protected by conservation laws.
\item E.g., in a finite-$T$ bath there is a characteristic relaxation time $\tau$ (and length $\ell$) after which a generic fluctuation gets re-absorbed by the bath.
\item At low energies, we only expect a handful of degrees of freedom to be relevant and long-lived (compared to $\tau$, e.g.), primarily (but not exclusively) those protected by conservation laws.
\item Local fluctuations of fields related to conserved quantities (like charges) cannot be absorbed, but only transported.  For such quantities, the transport mechanism typically takes a time $t_{\lambda}\gg \tau$ to return the system to equilibrium for a fluctuation of wavelength $\lambda \gg \ell$ (this double limit is also the typical regime studied in hydro).  These fluctuations are the interesting ones and are called the \textit{slow variables.}
\item In special situations, like near critical points, other \textit{slow variables} can also emerge, but these are non-generic.
\end{itemize}
In \cite{Glorioso:2018wxw}, our primary source, the EFT structure describing systems of the above type is explained.

\begin{minted}{python}
  def hello_world(args):
    print("hello world this is a test")
\end{minted}

\bibliographystyle{utphys}
\bibliography{bibliography}
\end{document}
